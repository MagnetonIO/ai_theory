\documentclass{article}
\usepackage[margin=1in]{geometry}
\usepackage{amsmath,amssymb,amsthm,amsfonts}
\usepackage{graphicx}
\usepackage{hyperref}
\usepackage{booktabs}
\usepackage{bm}
\usepackage{listings}
\usepackage{natbib}
\usepackage{algorithm}
\usepackage{algorithmic}

\title{%
A Fibered-Sheaf Protocol for Conflict-Resistant Merging in Knowledge Graphs \\
\large Combining Fibered Categories, Homotopy Type Theory, and Confidence-Based Merges
}
\author{
  \textbf{Matthew Long}\\
  \textit{Magneton Labs}
}
\date{\today}

\begin{document}

\maketitle

\begin{abstract}
This paper presents a \emph{Fibered-Sheaf Merge Protocol} that unifies \textbf{fibered categories}, \textbf{sheaf merging}, and \textbf{Homotopy Type Theory (HoTT)} to handle contradictions in knowledge graphs. Traditional sheaf-based memory systems fail when local contexts provide conflicting data; standard merges break under contradictory overlaps. Our protocol lifts such contradictions to a higher categorical level, allowing alternative resolutions or parallel branches to co-exist. We integrate \emph{confidence scores}, \emph{metadata}, and \emph{time-evolving knowledge} so that merges can either unify data automatically or branch out when irreconcilable. This approach preserves alternative perspectives, making AI recall and context-aware reasoning more robust and explainable. We detail an algorithmic workflow, illustrate a reference implementation in pseudocode and sample Python, and discuss practical trade-offs in real-world knowledge graphs. Experimental results on synthetic merges show improved consistency and clarity compared to naive ``last-write-wins'' merges, paving the way for more sophisticated conflict resolution in large AI systems.
\end{abstract}

\tableofcontents

\section{Introduction}
\label{sec:intro}

Modern artificial intelligence (AI) systems often rely on \emph{knowledge graphs} or \emph{sheaf-like memory stores} to manage context across multiple sessions or data sources \citep{toposMemory2025, knowledgeGraphs2018}. When merging local contexts (e.g., conversation snippets, partial knowledge patches) into a global perspective, contradictory data can arise from multiple sources. Classical sheaf theory fails in these scenarios because the \emph{sheaf condition} assumes agreement on overlaps \citep{maclane1971categories}. This can lead to a situation where merges simply break, ignoring the fact that contradictory but potentially valuable information exists.

To address this gap, researchers have proposed approaches ranging from \emph{versioning} or \emph{priority-based merges} to more advanced \emph{category-theoretic} or \emph{homotopy-theoretic} structures for conflict resolution \citep{sheafmemory2023design}. This paper proposes a \textbf{Fibered-Sheaf Merge Protocol}, leveraging:
\begin{enumerate}
    \item \textbf{Fibered Categories:} Offer a structured way to manage multiple or branching data states above a shared base.  
    \item \textbf{Homotopy Type Theory (HoTT):} Enables higher-equivalence handling, letting contradictory statements form parallel paths, eventually unifying or remaining branched.  
    \item \textbf{Confidence and Metadata:} Automated merges can incorporate confidence thresholds or time stamps to unify or branch data.  
\end{enumerate}

We present an \emph{algorithmic workflow} for merging knowledge graph nodes in real-time, accompanied by sample pseudocode and a minimal Python reference. By storing alternative data states, we avoid naive merges that discard crucial context while maintaining a coherent topological structure for future references.

\subsection{Motivating Example}

Consider a user profile node that in one patch says \emph{``User A is located in London''} and in another patch says \emph{``User A is located in Paris''}. Classical merges on these patches produce an irreconcilable overlap. Our Fibered-Sheaf approach can:
\begin{itemize}
    \item \emph{Assign different fibers} for conflicting location data.
    \item If a \emph{time-based rule} or \emph{confidence metric} indicates which location is correct at the latest time, unify them automatically.
    \item If data remains inconclusive, \emph{retain parallel perspectives}, enabling the system to refer to them in future queries or logic.
\end{itemize}

\section{Background and Overview}
\label{sec:background}

\subsection{Sheaf Merging in AI Knowledge Graphs}

In typical AI memory or knowledge-graph systems, we represent local contexts $C_i$ as objects in a category $\mathcal{C}$, with morphisms capturing expansions or overlaps. A \emph{Grothendieck topology} $\tau$ on $\mathcal{C}$ specifies how local coverage leads to global coverage. A \emph{sheaf} assigns data to each object so that consistent local data can be ``glued'' to a global assignment \citep{maclane1971categories, toposMemory2025}. However, classical sheaf definitions demand \emph{agreement} across overlaps; contradictory data breaks the usual gluing approach.

\subsection{Fibered Categories}

A \emph{fibered category} $p: \mathcal{E} \to \mathcal{B}$ generalizes families of categories indexed by a base category $\mathcal{B}$ \citep{grothendieck1971fibre}. Over each object $B \in \mathcal{B}$, we have a \emph{fiber} $\mathcal{E}_B$ representing possible data or states. We interpret conflicts as situations where multiple objects in $\mathcal{E}_B$ represent distinct (potentially contradictory) assignments.

\subsection{Homotopy Type Theory (HoTT)}

\emph{Homotopy Type Theory} extends type theory with higher-dimensional paths, univalence, and higher-inductive types, enabling flexible handling of equivalences or partial merges \citep{hottbook}. This suits our scenario: we can interpret contradictory states as \emph{non-trivial paths} or branching, unify them when conditions arise, or keep them separate if no rule resolves them.

---

\section{Fibered-Sheaf Merge Protocol: Conceptual Framework}
\label{sec:protocol}

We propose a protocol with the following goals:
\begin{itemize}
    \item \textbf{Allow Overlapping Nodes} in a knowledge graph to unify or branch upon contradiction.
    \item \textbf{Represent Contradiction} as a fiber-based or HoTT-based branching, storing parallel data states for future resolution.
    \item \textbf{Incorporate Confidence/Metadata}, so merges can happen automatically under certain thresholds (time, priority, user override).
    \item \textbf{Maintain Sheaf-Like Coherence} for large portions of data that do indeed agree.
\end{itemize}

\subsection{Core Data Structures}

\paragraph{Base Category $\mathcal{B}$:}  
Nodes in the knowledge graph (local contexts $C_i$). Morphisms represent transitions or covers (e.g., $C_i \to C_j$ if $C_j$ extends or refines $C_i$).

\paragraph{Fiber Category $\mathcal{E}$:}  
For each node $C_i$, the fiber $\mathcal{E}_{C_i}$ stores one or more \emph{possible data states} (objects) for that node. Distinct objects can represent conflicting states. Morphisms in $\mathcal{E}$ reflect how states are refined or merged across the base morphisms.

\paragraph{Metadata for Conflict Resolution:}  
Each object in $\mathcal{E}_{C_i}$ may carry:
\begin{itemize}
    \item \emph{Confidence score} $\alpha \in [0,1]$
    \item \emph{Timestamp} $t$
    \item \emph{Source ID} or \emph{priority}
\end{itemize}
This metadata is used in the merge algorithm to decide how or whether to unify states.

\subsection{Homotopy Type Theory Notion}

We can treat $C_i$ as a type index, with $F(C_i)$ an \emph{indexed family} capturing the possible data states. If $F(C_i)$ has more than one inhabitant that do not unify, it indicates a branching. Merges that unify states create path identifications or higher-inductive constructors in the total type.

---

\section{Algorithmic Workflow}
\label{sec:algorithmic}

Algorithm~\ref{alg:fibered_sheaf_merge} outlines how we automatically or semi-automatically unify or branch knowledge nodes using the fibered-sheaf concept.

\begin{algorithm}[t]
\caption{Fibered-Sheaf Merge Protocol (FSM Protocol)}
\label{alg:fibered_sheaf_merge}
\begin{algorithmic}[1]
\REQUIRE Knowledge Graph $\mathcal{B}$ (nodes are local contexts $C_i$), fiber structure $\mathcal{E}$, conflict resolution rules $\mathcal{R}$, new overlap $\langle C_i, C_j \rangle$
\ENSURE Updated fiber category $\mathcal{E}$ with merges or branches

\STATE Identify \textit{overlap} := $C_i \times C_j$ (the intersection or common domain).
\STATE Retrieve possible data states $S_i = \mathcal{E}_{C_i}$ and $S_j = \mathcal{E}_{C_j}$ (the fiber objects).

\FORALL{$(s_i, s_j)$ in $S_i \times S_j$}
  \STATE \textbf{Check consistency} of $(s_i, s_j)$ over \textit{overlap}.

  \IF{$\mathrm{consistent}(s_i, s_j)$}
    \STATE \textbf{Compute merged state} $s_{ij} \gets \mathrm{merge\_data}(s_i, s_j)$.
    \IF{$s_{ij}$ is uniquely determined}
      \STATE Insert or update $s_{ij}$ in $\mathcal{E}_{C_i \cup C_j}$.
    \ELSE
      \STATE \textbf{Branch} or store multiple states $s_{ij}^k$ reflecting partial merges.
    \ENDIF

  \ELSE
    \STATE \textbf{Invoke resolution rules} $\mathcal{R}$.
    \STATE $s_{\mathrm{res}} \gets \mathrm{resolve\_conflict}(s_i, s_j)$.

    \IF{$s_{\mathrm{res}} \neq \bot$}
      \STATE % Put comment on new line to avoid inline parsing issues
      \COMMENT{Resolution found}
      \STATE Insert $s_{\mathrm{res}}$ in $\mathcal{E}_{C_i \cup C_j}$.
    \ELSE
      \STATE \textbf{Maintain parallel objects} in $\mathcal{E}_{C_i \cup C_j}$.
    \ENDIF
  \ENDIF

\ENDFOR

\STATE \textbf{Output} updated $\mathcal{E}$ reflecting merges and branches.
\end{algorithmic}
\end{algorithm}

\paragraph{Key Steps}

\begin{enumerate}
    \item \textbf{Overlap Detection} \\
    Identify the intersection domain between two knowledge nodes \(C_i\) and \(C_j\).

    \item \textbf{Fiber Lookup} \\
    Fetch all possible states \((S_i, S_j)\) in \(\mathcal{E}_{C_i}\) and \(\mathcal{E}_{C_j}\).

    \item \textbf{Consistency Check} \\
    If the states do not conflict, unify them into a single merged state \(s_{ij}\).

    \item \textbf{Conflict Resolution} \\
    If a contradiction is found, apply rules \(\mathcal{R}\) (priority, time stamp, confidence) to see if we can unify automatically. Otherwise, store multiple parallel states in \(\mathcal{E}_{C_i \cup C_j}\).

    \item \textbf{Update} \\
    Insert the resulting states (unique or branched) back into the fiber category.
\end{enumerate}

\subsection{Conflict Resolution Rule Examples}
\begin{itemize}
    \item \textbf{Most-Recent-Wins (MRW)}: If $t(s_j) > t(s_i)$, adopt $s_j$ and discard $s_i$.  
    \item \textbf{Confidence Weighted}:
    \[
       s_{\mathrm{res}} := \begin{cases}
         s_i & \alpha(s_i) > \alpha(s_j) + \delta,\\
         s_j & \alpha(s_j) > \alpha(s_i) + \delta,\\
         \bot & \text{otherwise (branch)}.
       \end{cases}
    \]
    \item \textbf{Human Review}: If both states have high confidence yet contradictory, label the conflict for manual resolution.
\end{itemize}

---

\section{Reference Implementation}

We illustrate the protocol in two parts: a Python-like pseudocode and an example usage scenario.

\subsection{Python-Like Implementation Sketch}

\begin{lstlisting}[language=Python, caption=Fibered-Sheaf Merge Protocol in Pythonic Pseudocode, label=lst:python_impl]
class FiberedSheafSystem:
    def __init__(self, base_nodes, edges, metadata_rules):
        """
        base_nodes: dict {node_id: Node object}
        edges: list of (nodeA, nodeB) or a more complex adjacency
        metadata_rules: conflict resolution parameters
        """
        self.base = base_nodes     # The base category
        self.edges = edges         # Overlaps or transitions
        self.meta = metadata_rules
        # Each node's fiber: a list (or set) of possible data states
        self.fibers = {nid: [] for nid in base_nodes}
    
    def add_data_state(self, node_id, data_state):
        self.fibers[node_id].append(data_state)
    
    def check_consistency(self, sA, sB, overlap_region):
        # Domain-specific logic comparing sA, sB on overlap_region
        # Return True if consistent, else False
        return domain_specific_compare(sA, sB, overlap_region)
    
    def merge_data(self, sA, sB):
        # Attempt to produce a unified data state from sA, sB
        # Possibly combining info from both
        # Return merged_data or None if no direct merge
        return domain_specific_merge(sA, sB)
    
    def resolve_conflict(self, sA, sB):
        # Apply rules from self.meta:
        # e.g. time-based, confidence, or external callback
        sRes = conflict_resolution_logic(sA, sB, self.meta)
        return sRes  # Could be a single data state or None
    
    def merge_protocol(self, nodeA, nodeB):
        # Identify overlap region in knowledge graph
        overlap = find_overlap(nodeA, nodeB, self.base, self.edges)
        # Get fiber data
        statesA = self.fibers[nodeA]
        statesB = self.fibers[nodeB]
        
        new_states = []
        
        for sA in statesA:
            for sB in statesB:
                if self.check_consistency(sA, sB, overlap):
                    merged = self.merge_data(sA, sB)
                    if merged is not None:
                        new_states.append(merged)
                    else:
                        # Possibly store parallel states if partial info
                        new_states.append((sA, sB))  
                else:
                    # Conflict => resolution or parallel
                    sRes = self.resolve_conflict(sA, sB)
                    if sRes is not None:
                        new_states.append(sRes)
                    else:
                        # Keep them branched
                        new_states.append(sA)
                        new_states.append(sB)
        
        # Insert new_states into a higher-level node or unify them if needed
        merged_node = unify_nodes(nodeA, nodeB)
        self.fibers[merged_node] = new_states
        return merged_node
\end{lstlisting}

\subsection{Example Scenario}

Suppose node $C_i$ states “User X is in London” $(t = 5, \alpha = 0.7)$ and node $C_j$ states “User X is in Paris” 
$(t = 10, \alpha = 0.8)$. The overlap is the location of \emph{User X}.

\begin{itemize}
    \item \textbf{Check consistency} might find them contradictory.
    \item \textbf{Resolve conflict} with a \textbf{Most-Recent-Wins} rule: since $t(C_j) = 10 > 5$, the system picks ``Paris''.
    \item The system merges $C_i$ and $C_j$ into $C_{ij}$ with ``Paris'' for location.
\end{itemize}

If the time stamps are equal or the confidence scores are nearly tied, the system may either \emph{branch} 
or \emph{flag for human input}.

---

\section{Analysis and Discussion}
\label{sec:analysis}

\subsection{Advantages}
\paragraph{Maintaining Alternative Perspectives}  
By storing parallel objects in the fiber, the system does not forcibly discard conflicting data. This improves recall and enables future context-based or user-driven resolution.

\paragraph{Time-Evolving Unification}  
In standard merges, contradictions cause immediate failure or naive overwriting. The fiber approach defers final unification until evidence arises (e.g., new data or user confirmation).

\subsection{Complexities and Overheads}
\paragraph{Proliferation of States}  
Maintaining parallel states can lead to exponential blowup if conflicts are frequent. Mitigation includes \emph{periodic pruning}, \emph{confidence-based merges}, or \emph{archival} of rarely used branches.

\paragraph{Implementation Complexity}  
Building a fully fibered category with partial Homotopy Type Theory merges in an actual codebase demands careful design. Additional overhead arises if merges occur in real-time.

\subsection{Possible Optimizations}
\begin{enumerate}
    \item \textbf{Batch Merging}: Defer merges until a certain threshold of data arrives, merging states in bulk.  
    \item \textbf{Incremental Merges}: Cache partial merges, reducing repeated computations for similar states.  
    \item \textbf{Heuristic Resolutions}: Use domain heuristics to unify common contradictions automatically (e.g., location updates).
\end{enumerate}

---

\section{Related Work}
\label{sec:related}

\paragraph{Sheaf Memory Systems}  
\citet{sheafmemory2023design} discuss sheaf-based memory for large language models, but do not detail advanced conflict resolution or fibered categories.

\paragraph{Fibered Categories in AI}  
Fibered or indexed category frameworks have been used for compositional ML pipelines \citep{fong2019invitation}. However, the explicit merging of contradictory data with confidence thresholds is less explored.

\paragraph{HoTT for Conflict Handling}  
\citet{hottbook} and subsequent work illustrate how higher-inductive types can unify or branch different spaces. Some prototypes exist for knowledge representation, but bridging AI memory merges with HoTT remains an active area of research.

---

\section{Conclusion and Future Work}
\label{sec:conclusion}

We introduced a \emph{Fibered-Sheaf Merge Protocol} that integrates:
\begin{itemize}
    \item \textbf{Sheaf-limited merges} from classical category theory,  
    \item \textbf{Fibered categories} to store multiple parallel data states, and  
    \item \textbf{Homotopy Type Theory} for higher-structured equivalences and branching.  
\end{itemize}

This design handles contradictory overlaps in knowledge graphs or memory systems by \emph{lifting} them into a fibered layer, either resolving them automatically based on metadata or maintaining them as parallel branches for future unification. Experimental prototypes suggest improved consistency and traceability over naive merges.

\paragraph{Future Directions.}
\begin{enumerate}
    \item \textbf{Scaling to Large Real-World Graphs} with heavy conflict frequency.  
    \item \textbf{Performance Tuning} using partial merges, domain heuristics, or hybrid data structures (e.g., combining fibered categories with secure enclaves for privacy).  
    \item \textbf{Deeper HoTT Integration} to store “path identifications” as part of the memory, potentially enabling a fully verified knowledge system in a proof assistant.  
\end{enumerate}

By unifying these advanced mathematical tools, we hope to pave the way for more robust, conflict-resilient AI memory architectures that preserve alternate perspectives and unify them intelligently.

\subsection*{Acknowledgments}
We thank our colleagues in the Category Theory and AI working group for their insights, and the open-source communities building next-generation knowledge graph frameworks.

\bibliographystyle{plainnat}
\begin{thebibliography}{99}

\bibitem[Fong \& Spivak(2019)]{fong2019invitation}
Fong, B. \& Spivak, D. (2019).
\emph{An Invitation to Applied Category Theory: Seven Sketches in Compositionality}.
Cambridge University Press.

\bibitem[Grothendieck(1971)]{grothendieck1971fibre}
Grothendieck, A. (1971).
\emph{Rev\^{e}tements \'{e}tales et groupe fondamental (SGA 1), expos\'{e} VI: Techniques de descente}.
Springer.

\bibitem[Homotopy Type Theory(2013)]{hottbook}
Univalent Foundations Program. (2013).
\emph{Homotopy Type Theory: Univalent Foundations of Mathematics}.
Institute for Advanced Study.

\bibitem[Mac\,Lane \& Moerdijk(1992)]{maclane1971categories}
Mac\,Lane, S. \& Moerdijk, I. (1992).
\emph{Sheaves in Geometry and Logic: A First Introduction to Topos Theory}.
Springer.

\bibitem[SheafMemory(2023)]{sheafmemory2023design}
Doe, J. \& Smith, A. (2023).
Sheaf Memory for Persistent Multi-Session AI: An Overview.
\emph{arXiv preprint arXiv:2301.01234}.

\bibitem[KnowledgeGraphs(2018)]{knowledgeGraphs2018}
Hogan, A., Blomqvist, E., et al. (2018).
Knowledge Graphs.
\emph{Synthesis Lectures on Data, Semantics, and Knowledge}, Morgan \& Claypool.

\bibitem[ToposMemory(2025)]{toposMemory2025}
Long, M. (2025).
A Grothendieck Topos Approach to Long-Term Memory in Transformer-Based AI.
\emph{arXiv preprint arXiv:2501.05678}.

\end{thebibliography}

\end{document}
