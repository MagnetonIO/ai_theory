\documentclass[11pt]{article}
\usepackage[utf8]{inputenc}
\usepackage[T1]{fontenc}
\usepackage{geometry}
\geometry{margin=1in}
\usepackage{array}
\usepackage{tabularx}
\usepackage{amsmath}
\usepackage{hyperref}
\usepackage{booktabs}

\title{AI-Prompt-Driven Curriculum for Functorial Physics}
\date{}
\begin{document}

\maketitle

\noindent
\textbf{A modular, self-guided curriculum using AI prompts to explore and internalize the framework. At each step, feed the prompt to your favorite LLM, refine its output, and compare against canonical references.}

\vspace{1em}

\renewcommand{\arraystretch}{1.3}
\begin{tabularx}{\textwidth}{>{\raggedright\arraybackslash}p{1.8cm} >{\raggedright\arraybackslash}X >{\raggedright\arraybackslash}X}
\toprule
\textbf{Module} & \textbf{Learning Goals} & \textbf{AI Prompt} \\
\midrule
1. Foundations of Category Theory & Understand objects, morphisms, functors, natural transformations & ``Explain the basic concepts of category theory—objects, morphisms, functors, and natural transformations—with simple examples.'' \\
\midrule
2. Classical Mechanics as a Category & Model symplectic manifolds and canonical maps categorically & ``Describe the category whose objects are symplectic manifolds and whose morphisms are symplectomorphisms, including examples.'' \\
\midrule
3. Quantum Mechanics as a Category & View Hilbert spaces and unitaries in categorical terms & ``Define the category of Hilbert spaces with unitary morphisms. How do quantum channels generalize this?'' \\
\midrule
4. Building the Quantization Functor & Construct $\mathcal{Q} : \mathbf{C} \to \mathbf{Q}$ on simple systems & ``Show how to quantize the 1D harmonic oscillator by defining a functor from its phase space to a Hilbert space and symplectic flow to the unitary propagator.'' \\
\midrule
5. Semiclassical \& De-Quantization Functor & Capture $\hbar \to 0$ limits as a functor & ``Explain how to construct a functor from quantum systems back to classical phase spaces via WKB approximation, illustrating with coherent states.'' \\
\midrule
6. Adjointness \& Natural Transformations & Verify the adjoint relationship and correspondence principle & ``Demonstrate the unit and counit natural transformations that witness an adjunction between quantization and semiclassical functors.'' \\
\midrule
7. Applications \& Extensions & Explore field theories, gauge symmetry, and topological functors & ``Propose how higher-category or 2-functor structures can encode path integrals or gauge symmetry in Functorial Physics.'' \\
\midrule
8. Research Project & Design a small research question in functorial unification & ``Using Functorial Physics, outline a project to functorially quantize a simple gauge theory (e.g. U(1) electromagnetism) and compare classical and quantum structures.'' \\
\bottomrule
\end{tabularx}

\end{document}